
\chapter*{Introducción}  
\addcontentsline{toc}{chapter}{Introducción}

Las plantas son productores primarios que desempe\~nan un papel fundamental en la sostenibilidad de la vida en la Tierra \citep{mithofer2016general} y representan una rica fuente de nutrientes para muchos organismos, incluyendo: bacterias, hongos, protistas, insectos y vertebrados. Por su parte, los hombres dependen casi exclusivamente de las plantas para la alimentaci\'on y las utilizan, adem\'as, para la obtenci\'on de otros productos importantes como maderas, medicinas y biocombustibles \citep{freeman2008overview}.\\

Las plantas son organismos s\'esiles, por lo que est\'an obligados a discriminar entre los diferentes retos que les plantea su entorno y responder a ellos \citep{vivanco2005mecanismos}. Continuamente se enfrentan a condiciones adversas tanto bióticas, como abióticas, frente a las cuales se desencadenan respuestas de defensa. \\

Entre los mecanismos de defensa que se pueden desencadenar se incluye el aumento de la actividad de enzimas que participan en las respuestas de defensa, como la super\'oxido dismutasa (SOD), la catalasa (CAT) y la polifenol oxidasa (PPO).\\

La SOD es una familia de enzimas que constituye la primera l\'inea de defensa antioxidante en plantas, catalizando la descomposici\'on de una de las especies reactivas del ox\'igeno (ROS, por sus siglas en ingl\'es), el ani\'on super\'oxido. La actividad de la enzima le permite a la planta mantener un balance entre la producci\'on y eliminaci\'on de este ani\'on, que a altas concentraciones puede provocar da\~nos significativos en las estructuras celulares \citep{gill2015superoxide}.\\
 
La CAT es una hemoprote\'ina tetram\'erica que est\'a presente en la mayor\'ia de los organismos aerobios y es una de las enzimas principales en el catabolismo del per\'oxido de hidr\'ogeno \citep{luhova2003activities}. El per\'oxido de hidr\'ogeno participa en numerosos procesos de regulaci\'on dentro de la planta \citep{xie2019roles} y su balance es imprescindible, ya que puede ocasionar da\~nos oxidativos a estructuras celulares. Adem\'as, a partir del per\'oxido de hidr\'ogeno, en presencia de metales de transici\'on, se puede formar el radical hidroxilo que tiene la capacidad de reaccionar rápidamente con todo tipo de macromoléculas y no existe un mecanismo enzimático que lo elimine \citep{moller2007oxidative}.  \\

La PPO son una familia de metaloenzimas monom\'ericas que reaccionan con una amplia gama de sustratos fen\'olicos, catalizando la oxidaci\'on de fenoles propios de la c\'elula a quinonas. Han sido implicadas en las respuestas de defensa de las plantas debido a la aparici\'on de sus productos de reacci\'on en el momento de ataques de pat\'ogenos e insectos, lesiones y diferentes tipos de estr\'es \citep{mayer1979polyphenol, constabel1995systemin, maki2006development}. \\

Los diferentes mecanismos de defensa de las plantas frente a condiciones adversas son regulados mediante fitohormonas \citep{verma2016plant}. Los brasinoesteroides (BRs) pertenecen a este grupo de mol\'eculas y son esenciales en los procesos de expansi\'on y divisi\'on celular, diferenciaci\'on de tejidos, reproducci\'on y resistencia a distintos tipos de estr\'es bi\'oticos y abi\'oticos \citep{belkhadir2006brassinosteroid}. La aplicaci\'on ex\'ogena o la modificaci\'on de su contenido en plantas, puede conducir al incremento de la tolerancia al estr\'es de los cultivos \citep{moreno2018silico}, modulando la actividad de enzimas de defensa, entre ellas: SOD, CAT y PPO \citep{fariduddin2014brassinosteroids}, por lo que presentan un gran potencial en la agricultura \citep{hernandez2016brasinoesteroides}.\\ 

Entre los BRs, la Brasin\'olida (BL) presenta mayor actividad. Desafortunadamente su contenido en fuentes naturales es sumamente bajo y tanto su aislamiento como su s\'intesis, son costosas. La 24-epibrasin\'olida (24-epiBL), es un estereois\'omero de BL y es el BRs m\'as ampliamente utilizado hasta la fecha, pero sus aplicaciones prácticas son limitadas debido a su elevado precio. Por esta raz\'on, el desarrollo nuevas moléculas con buena actividad y bajo costo es de gran significado práctico \citep{lei2017structure}.\\

El laboratorio de Bioproductos del Centro de Investigación de Productos Naturales (CNPR) de la Facultad de Química de la Universidad de la Habana, desarroll\'o productos a partir de saponinas y fitoesteroles de plantas, y mediante ensayos de actividad biológica observaron que algunos de estos tienen efectos naturales similares a BRs. \cite{moreno2018silico} analizaron, mediante m\'etodos computacionales, la afinidad y las formas de uni\'on de estos compuestos sint\'eticos an\'alogos de BRs a su receptor BRI1 e indicaron que 17 de ellos pueden ser considerados buenos candidatos para la realizaci\'on de pruebas biol\'ogicas. \\

Dos de estos compuestos (DI-31 y MH-5) se seleccionaron para el an\'alisis del efecto de su aplicaci\'on en las enzimas de defensa SOD, CAT y PPO, en plantas de \textit{Raphanus sativus}. Para dise\~nar un protocolo que permita la evaluaci\'on de los efectos de estos an\'alogos sint\'eticos de brasinoesteroides, en la actividad de estas enzimas que participan en la defensa de las plantas, se hace necesario optimizar t\'ecnicas y metodolog\'ias para determinar la actividad de estas enzimas. 


%Antecedentes, problema, objetivo

%Los BRs est\'an involucrados en la regulaci\'on del metabolismo de las especies reactivas del ox\'igeno (ROS, por sus siglas en ingl\'es) mediante la expresi\'on de genes antioxidantes, los cuales incrementan la actividad de enzimas antioxidantes \citep{cao2005loss, ogweno2008brassinosteroids}.\\








\chapter*{Objetivo General}  
\addcontentsline{toc}{chapter}{Objetivos}

Dise\~{n}ar protocolos para determinar la actividad de enzimas de defensa en plantas de \textit{Raphanus sativus}.\\

\section*{Objetivos específicos}

\begin{enumerate}
\item Determinar una metodolog\'ia para la obtenci\'on del extracto crudo de prote\'inas.
\item Establecer los ensayos que permitan determinar los par\'ametros cin\'eticos de las enzimas super\'oxido dismutasa (SOD), catalasa (CAT) y polifenol oxidasa (PPO).
\end{enumerate}

%Describir, seleccionar, identificar, determinar, dise\~nar, establecer, 