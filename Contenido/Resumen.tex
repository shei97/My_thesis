{\bf \LARGE Resumen}

\medskip
\medskip
\medskip
\medskip
\medskip
\medskip

Las plantas son organismos sésiles que se enfrentan a condiciones adversas tanto bióticas, como abióticas. Ante estas condiciones se desencadenan respuestas de defensa que incluyen el aumento de la actividad de un grupo de enzimas. Los diferentes mecanismos de defensa son regulados mediante fitohormonas, entre ellas los brasinoesteroides (BRs), cuya aplicación exógena o la modificación de su contenido en plantas, puede conducir al incremento de la actividad de estas enzimas. Este trabajo tiene como objetivo diseñar un protocolo que permita la posterior evaluación del efecto de dos análogos sintéticos de BRs en las enzimas de defensa superóxido dismutasa (SOD), catalasa (CAT) y polifenol oxidasa (PPO) en plantas de \textit{Raphanus sativus}. En el trabajo se identifican las metodologías para la preparación de extractos enzimáticos que permitan evaluar la actividad de las mismas. Para la medición de SOD se establece la metodología de extracción con acetona como correcta, mientras que las enzimas CAT y PPO se evalúan en extractos acuosos. Se establecen los ensayos que permiten la determinación de los parámetros cinéticos de las enzimas SOD, CAT y PPO en extractos enzimáticos crudos.


\medskip
\medskip
\medskip
\medskip
\medskip
\medskip
\medskip
\medskip
\medskip
\medskip
\medskip
\medskip
\medskip
\medskip

{\bf Palabras clave:} catalasa,  extractos enzim\'aticos crudos, polifenol oxidasa, \textit{Raphanus sativus}, super\'oxido dismutasa.


